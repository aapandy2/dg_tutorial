%// Distributed under the MIT License.
%// See LICENSE.txt for details.

\documentclass[12pt]{article}
\usepackage{graphicx}
\usepackage{amsmath}
\usepackage{amsthm}
\usepackage{amssymb}
\usepackage{color}
\usepackage{braket}
\usepackage{bm}
\usepackage[margin=1in]{geometry}
\usepackage{mathtools}
\usepackage{tikz}
\usepackage{listings}

\allowdisplaybreaks
\numberwithin{equation}{section}

\interfootnotelinepenalty=10000

\usepackage{calligra}
\PassOptionsToPackage{hyphens}{url}\usepackage{hyperref}
\hypersetup{colorlinks=true, linkcolor=blue, citecolor=violet}

\DeclareMathAlphabet{\mathcalligra}{T1}{calligra}{m}{n}
\DeclareFontShape{T1}{calligra}{m}{n}{<->s*[2.2]callig15}{}
\newcommand{\scriptr}{\mathcalligra{r}\,}
\newcommand{\boldscriptr}{\pmb{\mathcalligra{r}}\,}

\newcommand{\Lagr}{\mathcal{L}}
\newcommand{\Hami}{\mathcal{H}}
\newcommand{\reals}{\rm I\!R}
\newcommand{\order}{\mathcal{O}}
\newcommand{\bx}{\mathbf{x}}
\newcommand{\bp}{\mathbf{p}}
\newcommand{\bq}{\mathbf{q}}
\newcommand{\redtext}[1]{\textcolor{red}{#1}}
\newcommand{\pvec}[1]{\vec{#1}\mkern2mu\vphantom{#1}}
\newcommand{\mA}{\mathcal{A}}

%\setcounter{section}{-1}

\begin{document}
\title{DG Tutorial}
\author{Alex Pandya}
\date{\today}
\maketitle
%\tableofcontents
%\clearpage

\section{DG for the 1D scalar advection equation}

We are interested in solving conservation laws of the form
\begin{equation}
\partial_t \bm{q} + \partial_i \bm{f}^i = \bm{S},
\end{equation}
but here we will consider the simplest possible case: the 1D scalar advection equation,
\begin{equation}
\partial_t q + \partial_x (c q) = 0.
\end{equation}

In order to derive the DG discretization, we multiply through by a smooth, compactly supported test function $\phi$ and integrate over the solution domain.
To simplify things later on, we'll divide the (1D) domain into $k$ cells of extent $\Omega_k \equiv x \in [x_L, x_R]$:
\begin{equation} \label{eq:weak_form}
\begin{aligned}
0 &= \int_{\Omega_k} \phi \, \partial_t q \, dx + \int_{\Omega_k} \phi \, \partial_x (c q) \, dx \\
&= \int_{x_L}^{x_R} \phi \, \partial_t q \, dx + \big[ c q \phi \big]^{x_R}_{x_L} - \int_{x_L}^{x_R} (c q) \, \partial_x \phi \, dx. \\
\end{aligned}
\end{equation}
Note that a weak solution satisfies the above integral equation \textit{for all admissible $\phi$,} and the global solution satisfies it for all cells $k$.


We now express the spatial dependence of the solution $q$ in terms of basis functions $\psi(\xi)$ from a complete vector space $V$:
\begin{equation}
q(t,x) = \sum_{i=0}^{\infty} a_i(t) \, \psi_i\big[ \xi(x) \big],
\end{equation}
where we have allowed the basis functions $\psi_i$ to be functions of a different coordinate $\xi$ which is a function of $x$.

We now approximate the solution by taking only the first $M$ basis functions:
\begin{equation}
\tilde{q}(t,x) = \sum_{i=0}^{M} a_i(t) \, \psi_i\big[ \xi(x) \big].
\end{equation}

Given that the weak form of the conservation law (\ref{eq:weak_form}) must be satisfied \textit{for all $\phi$}, we can express $\phi$ in terms of the same basis functions $\psi_i$, and since $V$ is complete, we can express any arbitrary $\phi$ as a linear combination of the $\psi_i$.  The second DG approximation comes in that we take only the first $M$ basis functions for $\phi$ as well, so we define
\begin{equation}
\phi_i \equiv \psi_i
\end{equation}
and we get a separate copy of (\ref{eq:weak_form}) for each $j \in [0, M]$:
\begin{equation} \label{eq:weak_form_approx}
\begin{aligned}
0 &= \int_{x_L}^{x_R} \phi_j \, \partial_t \tilde{q} \, dx + \big[ f(\tilde{q}) \, \phi_j \big]^{x_R}_{x_L} - \int_{x_L}^{x_R} (c \tilde{q}) \, \partial_x \phi_j \, dx \\
&= \partial_t \sum_{i=0}^{M} a_i(t) \int_{x_L}^{x_R} \phi_j\big[ \xi(x) \big] \psi_i\big[ \xi(x) \big] \, dx + \Big[ f(\tilde{q}) \, \phi_j\big[ \xi(x) \big] \Big]^{x_R}_{x_L} \\
&- c \sum_{i=0}^{M} a_i(t) \int_{x_L}^{x_R} \psi_i\big[ \xi(x) \big] \partial_x \phi_j\big[ \xi(x) \big] \, dx \\
&= \partial_t \sum_{i=0}^{M} a_i(t) \int_{\xi(x_L)}^{\xi(x_R)} \phi_j(\xi) \psi_i(\xi) \, \frac{\partial x}{\partial \xi} \, d\xi + \Big[ f(\tilde{q}) \, \phi_j\big[ \xi(x) \big] \Big]^{x_R}_{x_L} \\
&- c \sum_{i=0}^{M} a_i(t) \int_{\xi(x_L)}^{\xi(x_R)} \psi_i(\xi) \Big[ \frac{\partial \xi}{\partial x} \partial_\xi \phi_j(\xi) \Big] \, \frac{\partial x}{\partial \xi} d \xi \\
&= \partial_t \sum_{i=0}^{M} a_i(t) \int_{\xi(x_L)}^{\xi(x_R)} \phi_j(\xi) \psi_i(\xi) \, J \, d\xi + \Big[ f(\tilde{q}) \, \phi_j\big[ \xi(x) \big] \Big]^{x_R}_{x_L} \\
&- c \sum_{i=0}^{M} a_i(t) \int_{\xi(x_L)}^{\xi(x_R)} \psi_i(\xi) \partial_\xi \phi_j(\xi) \, d \xi, \\
\end{aligned}
\end{equation}
where we have defined the \textit{jacobian of the coordinate transformation}
\begin{equation}
J \equiv \frac{\partial x}{\partial \xi}.
\end{equation}
Notice that the factor of the jacobian cancels from the last term above, because the spatial derivative introduces a factor of $J^{-1}$ to cancel it.

We now define
\begin{equation}
\begin{aligned}
M_{ij} &\equiv \int_{\xi(x_L)}^{\xi(x_R)} \phi_j(\xi) \psi_i(\xi) \, J \, d\xi \\
K_{ij} &\equiv \int_{\xi(x_L)}^{\xi(x_R)} \psi_i(\xi) \partial_\xi \phi_j(\xi) \, d \xi \\
f_j    &\equiv \Big[ f(\tilde{q}) \, \phi_j\big[ \xi(x) \big] \Big]^{x_R}_{x_L} \approx \Big[ \hat{f}(\tilde{q}) \, \phi_j\big[ \xi(x) \big] \Big]^{x_R}_{x_L}
\end{aligned}
\end{equation}
which are known as the \textit{mass matrix}, \textit{stiffness matrix}, and \textit{flux (boundary) term}, respectively. 
Notice that we have inserted a \textit{numerical flux function}, $\hat{f}$, in for the continuum flux; the choice of this function will determine how well our numerical scheme can handle discontinuous solutions.

Using these definitions we can write (\ref{eq:weak_form_approx}) as a discrete approximation to the PDE:
\begin{equation} \label{eq:discrete_approx}
\begin{aligned}
0 &= \sum_{i=0}^{M} \partial_t a_i(t) M_{ij} + \Big[ \hat{f}(\tilde{q}) \, \phi_j\big[ \xi(x) \big] \Big]^{x_R}_{x_L} - c \sum_{i=0}^{M} a_i(t) K_{ij} \\
&= \bm{M} \cdot \partial_t \bm{a} + \bm{f} - c \bm{K} \cdot \bm{a} \\
\implies \partial_t \bm{a} &= \bm{M}^{-1} \cdot \Big( c \bm{K} \cdot \bm{a} - \bm{f} \Big), \\
\end{aligned}
\end{equation}
which gives the time evolution of the coefficient vector $a_j(t) = \bm{a}$.
Given a choice for the numerical flux function $f(\tilde{q})$, the basis functions $\psi(\xi), \phi(\xi)$, and the coordinate map $\xi(x)$, we can construct a discrete approximation to the PDE.

%\clearpage

\subsection{Simple example: $M=1$, $\psi_i$ polynomials}

We will now restrict to $M = 1$ with $\psi_i$ chosen to be polynomials.
Specifically, we choose $\hat{\psi}_0 = 1, \hat{\psi}_1 = \xi$, and then change to a linear combination of these basis vectors (for simplicity later on) where:
\begin{equation}
\psi_0 = 1 - \xi, ~~~ \psi_1 = \xi,
\end{equation}
defined on $\xi \in [0,1]$.
We'll take our spatial domain to be $x \in [0, 1]$ as well, but we'll divide the domain into $K$ cells so that each cell has extent $[x_L, x_R] = [k (\Delta x), (k+1) (\Delta x)]$ and $k$ is an integer with $k \in [0,K]$.
Hence in the $k$th cell our coordinate map is
\begin{equation}
x(\xi) = x_L + \xi (x_R - x_L) = k (\Delta x) + \xi (\Delta x),
\end{equation}
and the Jacobian is
\begin{equation}
J \equiv \frac{\partial x}{\partial \xi} = \Delta x,
\end{equation}
which is constant within a given simulation (assuming no grid refinement).

We can now compute the mass matrix, which is
\begin{equation}
M_{ij} = \Delta x \int_0^1 \phi_j(\xi) \psi_i(\xi) \, d\xi = \frac{\Delta x}{6}
\begin{pmatrix}
2 & 1 \\
1 & 2
\end{pmatrix}.
\end{equation}
The stiffness matrix is
\begin{equation}
K_{ij} = \int_0^1 \psi_i(\xi) \big[ \partial_\xi \phi_j(\xi) \big] \, d\xi = \frac{1}{2}
\begin{pmatrix}
-1 & 1 \\
-1 & 1
\end{pmatrix}.
\end{equation}
To evaluate the flux term, we have to specify the numerical flux function.
Here we will take the upwind flux, which in the $k$th cell takes the form
\begin{equation}
\hat{f}\big[ \tilde{q}_{k}(x_L) \big] =
\begin{pmatrix}
c \tilde{q}_{k-1}(x_R) & c > 0 \\
c \tilde{q}_{k}(x_L)   & c < 0,
\end{pmatrix}, ~~~
\hat{f}\big[ \tilde{q}_{k}(x_R) \big] =
\begin{pmatrix}
c \tilde{q}_{k}(x_R) & c > 0 \\
c \tilde{q}_{k+1}(x_L)   & c < 0,
\end{pmatrix}
\end{equation}
or in other words, if the flow has positive speed (i.e., the flux is to the right), then we use the \textit{upwind value} at the left interface, that is, the flux leaving the cell to the left.
Hence the flux term is (here we will take $c > 0$):
\begin{equation}
\begin{aligned}
f_j &= \Big[ \hat{f}(\tilde{q}) \, \phi_j\big[ \xi(x) \big] \Big]^{x_R}_{x_L} \\
&= \hat{f}\big[ \tilde{q}(x_R) \big] \, \phi_j\big[ \xi(x_R) \big]  - \hat{f}\big[ \tilde{q}(x_L) \big] \, \phi_j\big[ \xi(x_L) \big] \\
&= c \tilde{q}_{k}(x_R) \, \phi_j\big[ \xi(x_R) \big]  - c \tilde{q}_{k-1}(x_R) \, \phi_j\big[ \xi(x_L) \big] \\
&=
\begin{pmatrix}
c \tilde{q}_{k}(x_R) \, \phi_0(1)  - c \tilde{q}_{k-1}(x_R) \, \phi_0(0) \\
c \tilde{q}_{k}(x_R) \, \phi_1(1)  - c \tilde{q}_{k-1}(x_R) \, \phi_1(0)
\end{pmatrix} \\
&=
\begin{pmatrix}
- c \tilde{q}_{k-1}(x_R) \\
c \tilde{q}_{k}(x_R)
\end{pmatrix}
\end{aligned}
\end{equation}

\subsection{Less simple example: Legendre polynomials}

One is likely interested in a higher-order solution approximation within a cell than the linear one given in the previous section.
It turns out that naively using higher order monomials --- that is, a basis of the form 
\begin{equation}
\tilde{q} \sim \sum_{i=0}^M a_i x^i
\end{equation}
--- is unstable to Runge's phenomenon for $n$ larger than a few\footnote{talk about the ill-conditioned mass matrix? or just link out the description} (which is why we used $M=1$ in the previous example).
The basis of monomials is also not orthogonal; if they were, the mass matrix would be diagonal, which is clearly a beneficial property to have since we have to invert the mass matrix to obtain our solution.

For this reason, one typically instead uses the \textit{Lagrange basis}, taking for our basis functions
\begin{equation}
\ell_{q}(x) = \prod_{i=0, i \neq j}^{M} \frac{x - x_i}{x_j - x_i}.
\end{equation}
Note that by construction $\ell_q(x)$ has roots at the points $x_i$.
We can then use the functions $\ell_{q}(x)$ to construct an approximation to a function $f(x)$ given a set of $x$ values $x_i$ and their corresponding $y$ values $y_i = f(x_i)$:
\begin{equation}
L(x) = \sum_{i=0}^{M} a_i \ell_i(x).
\end{equation}



\end{document}
